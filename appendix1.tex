\chapter{\label{a:hints}Hints}

\begin{itemize}

    \item Learn about running latex and bibtex -- how many times, and in what
        order!  Or, if it is available for your operating system, try using
        \texttt{latexmk}.

    \item Examine the \texttt{.tex} files that created the present document, to
        see how things are done.  It is convenient to use the \texttt{include}
        syntax for chapters, so you can check formatting quickly, without
        processing the entire thesis.

    \item Use \texttt{bibtex} for your literature citations.  It may seem
        easier to simply enter your reference list in the text, but that is a
        serious mistake, because it leads to errors, and it forces you to
        reformat everything for particular journals.

    \item Put \texttt{label} commands inside figure captions.  Otherwise
        (sometimes) cross-referencing will be mixed up.

    \item Use \texttt{label} on any equations, figure, section, etc., that you
        may need to refer to later.  This gives flexibility in adding and
        removing material, and prevents errors.  Many people find it helpful to
        establish a convention for referencing items, e.g. perhaps ``e:...''
        for equations, ``f:...'' for figures, ``t:...'' for tables, ``c:...''
        for chapters.

    \item Use \emph{short} entries in the square-bracket part of captions for
        figures and tables.  This way, the table of contents will be in a form
        that is more likely to be approved by the Faculty of Graduate Studies.
        (Do this work at the start, because it is a pain to do this in your
        last few days at the university.)

    \item If you see unexpected cross-referencing results, delete your
        \texttt{*.aux} (and possibly other files, such as \texttt{.blg}), and
        try again.

    \item Graphics can be tricky. Do some tests early on, so you'll learn
        whether your LaTeX setup handles the graphs you make.  (For example, if
        you use the \texttt{R} language, you will probably want to use
        \texttt{pdf()} for line graphs and \texttt{png()} for image graphs.
        For the latter, try e.g.  \texttt{png("file.png", width=7, height=7,
        unit="in", res=300)}, which produces plots that have similar geometries
        to the pdf default.)

    \item Be careful with quotation marks. In latex the \verb|"|-key (i.e. the
        unidirected double quote character) produces double right quotation
        marks, and should never be used. Double quotation marks are produced by
        typing \verb|``| and \verb|''|. Most latex-oriented text editors will
        do the substitution for you.

    \item Find out what thesis-formatting rules will apply to you, and handle
        them as best you can.  For example, as of 2012, FGS is requesting that
        section titles be in title case, which means that words in titles are
        to start with upper-case letters, except for small words such as ``on''
        or ``the''.  So, bear this in mind while writing the section titles, to
        save time in editing later on.

\end{itemize}
